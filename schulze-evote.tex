\documentclass{llncs}
\usepackage{amsmath,amsfonts,wrapfig,graphicx,caption,url}
\usepackage{subcaption}
\usepackage[all]{xy}
\captionsetup{compatibility=false}
\pagestyle{plain}

\newcommand{\Nat}{\mathbb{N}}
\newcommand{\Z}{\mathbb{Z}}
\newcommand{\st}{\mathsf{st}}
\newcommand{\LFP}{\mathsf{LFP}}
\newcommand{\Pow}{\mathsf{Pow}}
\newcommand{\GFP}{\mathsf{GFP}}

\title{Homomorphic Schulze Counting}
\begin{document}
\maketitle

\section{Problem.}

Given a data structure that represents encrypted ballots, to compute
the margin function. There is choice in the data structure that we
use to represent ballots.

\section{Solutions}
\subsection{Vanessa et al's protocol}

This is taken from Appendix B of the shuffle sum paper. Given: the
ballot in preference order. Produces: an encrypted binary matrix
that indicates pairwise preferences for candidates.

\begin{itemize}
\item the preference order ballot (with encrypted candidate numbers)
\item tuples (E(0), *, *) and (E(-1), *, *) where * are suitably chose
\item candidates
\item a shuffle of these types
\item a ZKP that the shuffle is genuine
\item the shuffled tuples, where the 2nd and 3rd component are decrypted
\item evidence that the decryption of the 2nd and 3rd component of the
      tuple is actually correct
\item the matrix constructed from the tuples with decrypted 2nd and 3rd
      entry.
     
      
\end{itemize}
\begin{itemize}
\item Ballot representation (List representation, Memory requirement O(n))
1 $=>$ E(c$_{1}$), 2 $=>$ E(c$_{2}$), 3 $=>$ E(c$_{3}$), ..... n $=>$ E(c$_{n}$)
\item Tuple or matrix construction (Memory requirement O($n^2$)) 
\[\begin{bmatrix}
    (E(c_{1}), E(c_{1}), E(0)) & (E(c_{1}), E(c_{2}), E(1)) & (E(c_{1}), E(c_{3}), E(1)) & \dots  & (E(c_{1}), E(c_{n}), E(1)) \\
    (E(c_{2}), E(c_{1}), E(0)) & (E(c_{2}), E(c_{2}), E(0)) & (E(c_{2}), E(c_{3}), E(1)) & \dots  & (E(c_{2}), E(c_{n}), E(1)) \\
    \vdots & \vdots & \vdots & \ddots & \vdots \\
    (E(c_{n}), E(c_{1}), E(0)) & (E(c_{n}), E(c_{2}), E(0)) & (E(c_{n}), E(c_{3}), E(0)) & \dots  & (E(c_{n}), E(c_{n}), E(0))
\end{bmatrix}
\]
 \item The above matrix can be also thought of List containing only those pairs whose value is E(1). [(E(c$_{1}$), E(c$_{2}$), E(1)), (E(c$_{1}$), E(c$_{3}$), E(1)), ...... (E(c$_{n-1}$), E(c$_{n}$), E(1))]
 \item Chose a random permutation and shuffle the matrix constructed above 
 (each row and column) or List depending on data structure
 \item Zero knowledge proof that shuffle is genuine (data structure of ZKP)
 \item Decrypt first and second element (encryption of candidate), but not preference
 \item For evidence, we can publish the ephemeral key used in encryption. My understanding of this step is that voter can re-encrypt the candidates and match it to the encrypted ballots published on bulletin board, but he can't prove it to any external party that he voted for some particular candidate because he has no way to show the encrypted choice other than himself. 
 \item Matrix constructed by decrypting candidates (Memory O($n^2$))
 \item This scheme involved, encrypted ballot (Linear data O(n)), Tuple construction
 of encrypted candidate O($n^2$), random permutation (Linear data structure, but not
 published O(n)), Zero knowledge proof data structure, and final decrypted candidate 
 Matrix 
 O($n^2$)
 
\end{itemize}




\subsection{Making the matrix representation the main data
structure}

Given: the matrix of encrypted pairwise preferences (as main data
structure). Produces: evidence that the matrix is in fact an
encoding of preferences.

\begin{itemize}
\item  evidence that the matrix is binary, i.e. just has 0/1 entries
 using a shuffle + ZKP + decryption
\item  a shuffle of the rows + ZKP
\item  the homomorphic sums of the rows + decryption
\item  homomorphic sum of the diagonal, and diagonally symmetric elements
 + decryption.
\end{itemize}

The idea here is that a matrix which 
\begin{itemize}
\item is binary, i.e. just 0 and 1 entries
\item has zeros in the diagonal
\item the sum $a_{ij} + a_{ji} = 1$
\item has one row summing up to $n-1$, another row to $n-2$ and so
forth
\end{itemize}
is necessarily obtained from assigning preferences.

\begin{itemize}
\item We have encrypted pairwise preference (memory O($n^2$))
\item 
\end{itemize}


\subsection{Using a permutation Matrix}

Another idea is that one could represent a ballot by a permutation
matrix, but it's not clear how
\begin{itemize}
\item evidence that an encrupted matrix is really a permutation
matrix
\item given that we know it's a permutation matrix, how to compute
the pairwise preferences.
\end{itemize}

\subsection{Using Shuffle Naively}

This may be too naive, but how about the following: given a ballot
in candidate order
\begin{itemize}
\item decrypt, and construct the matrix of pairwise preferences
\item encrypt the pairwise preference matrix
\item pick a random permutation $\sigma$ 
\item shuffle the ballot using $\sigma$
\item shuffle the encrypted pairwise preference matrix using
$\sigma$ (both rows and columns)
\item decrypt the shuffled ballots + the shuffled preference matrix
and give evidence for the fact that they are indeed decrypted.
\end{itemize}

\begin{itemize}
\item Ballot representation (List representation, Memory requirement O(n))
c$_{1}$ $=>$ E(p$_{1}$),  c$_{2}$ $=>$ E(p$_{2}$), c$_{3}$ $=>$ E(p$_{3}$), ..... c$_{n}$ $=>$ E(p$_{n}$)
\item Decrypt ballot and construct margin matrix (memory O($n^2$))
\item Encrypt the matrix constructed in previous step 
\item Pick a random permutation  $\sigma$  (Memory O(n))
\item Shuffle the ballot using $\sigma$ (Memory O(n))
\item Do we need zero knowledge proof of shuffling ?
\item Shuffle the encrypted matrix using the $\sigma$ (row and column) (memory O($n^2$))
\item Decrypt shuffled ballots and shuffled preference matrix and evidence of 
correctness
\item The evidence here is ephemeral key. Assuming that each voter has
knowledge about their original vote (in plain text) and we have published the 
plain text shuffled vote, they can compute $\sigma$. By applying $\sigma^{-1}$ on 
decrypted shuffled preference matrix and encrypting the values using ephemeral key
they can compute the encrypted preference matrix and compare it our (server) 
encrypted preference matrix. 
\item What about coercion because each voter can prove his vote ?  
\item The data needed in this method is ballot representation in encrypted preference order (O(n)), decrypted shuffled ballot 
using permutation $\sigma$ (O(n)), Zero knowledge proof of shuffle, Encrypted preference matrix constructed from original ballot O($n^2$), Decrypted preference matrix constructed O($n^2$). 

	 


\end{itemize}

Remark from Jacub: can chose the permutation so that the pairwise
preference matrix is in upper diagonal form (ie. all ones above the
diagonal) so no need to publish the decrypted matrix. 

Even easier:
\begin{itemize}
\item publish encrypted pairwise preference matrix
\item pick a permutation that brings it in upper triangular form
(always possible if ballot is genuine)
\item publish encrypted permuted matrix + evidence that it's a
genuine permutation
\item publish evidence that it is in upper triangular form.
\end{itemize}

\subsection{Jakub's 2nd Solution}

Given: candidate order ballot
\begin{enumerate}
\item publish candidate order ballot: $(c_1, e(p_1), \dots (c_n,
e(p_n))$
\item publish candidate order ballot with encrypted preferences
\item shuffle the last ballot using specific permutation $\sigma$ so that encypted preferences are
ascending
\item publish this last ballot with decrupted preferences.
\item []
\item publish the matrix pairwise preference ballot, indexed by
  candidates $c_1, \dots c_n$ (in ascencing order) and encrypted
  preferences.
\end{enumerate}

To verify the latter, need to show that a shuffle by $\sigma^{-1}$
decrypts to an upper triangular matrix.

\subsection{Using Range Proofs}

Given: the encrypted ballot in candidate order.

\begin{itemize}
\item using homomorphic subtraction, construct the (encrypted)
matrix $m_{ij} = \mbox{pref. of cand. $a$ - pref. of candidate
$b$}$.
\item using decryption, construct the encrypted pairwise preference
matrix $(p_{ij})$
\item publish a shuffle of $(m_{ij})$ and $(p_{ij})$ plus evidence of correctness
of both shuffles
\item for every entry $m_{ij}$, use a range proof that witnesses
that $p_{ij}$ is $+1$ if $m_{ij}$ is positive, $0$ if $m_{ij}$ is
$0$ and $-1$, otherwise.
\end{itemize}
\end{document}




\end{document}
